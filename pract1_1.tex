\documentclass{article}
\usepackage[utf8]{inputenc}
\usepackage[english,russian]{babel}  % добавили русский
\usepackage{tikz}                    % черчение графиков
\usepackage{siunitx}                 % в электрических схемах номиналы элементов
\usepackage[american,cuteinductors,smartlabels]{circuitikz} % черчение электрических схем


\title{power\_electronics\_pract1} % знак подчеркивание - символ построчного индекса, замаскируем его обратным слешом
\author{taybola }
\date{April 2020}

\begin{document}

\maketitle

\section{Introduction}

\begin{circuitikz}
	\draw (0,0) -- (7,0); % отрезок, в будущем ось X
\end{circuitikz}

\begin{circuitikz}
\draw[thin,->] (0,0) -- (7,0); %сделаем ось красивее
\draw[thin,->] (0,0) -- (0,2); % ось y
\end{circuitikz}


\begin{circuitikz}
	\draw[thin,->] (0,0) -- (7,0) node[below] {$\omega t$}; %подпишем ось, гречечсие буквы как команды \alpha
	\draw[thin,->] (0,0) -- (0,2) node[left] {$U$} ; % ось y
\end{circuitikz}

\begin{circuitikz}
        \draw[thin,->] (0,0) -- (7,0) node[below] {$\omega t$}; %подпишем ось, гречечсие буквы как команды \alpha
        \draw[thin,->] (0,0) -- (0,2) node[left] {$U$} ; % ось y

	\draw[domain=0:4]   % чертим в области \x от 0 до 4 
	plot ( \x, 0.5*\x );   % \x - это переменная, 0.5*\x - это y(x), т.е. график в виде набора координат ( x, y)
\end{circuitikz}


\begin{circuitikz}
        \draw[thin,->] (0,0) -- (7,0) node[below] {$\omega t$}; %подпишем ось, гречечсие буквы как команды \alpha
        \draw[thin,->] (0,0) -- (0,2) node[left] {$U$} ; % ось y

        \draw[domain=0:6]
	plot ( \x, {sin(\x r)} );   % сложные математические выражения приходится заключать в фигурные скобки, r - аргумент в радианах (по-умолчанию в градусах) 
\end{circuitikz}

\end{document}

