\documentclass[a4paper,12pt]{article}
\usepackage{extsizes}

%\usepackage[T2A]{fontenc}
%\usepackage[utf8x]{inputenc}
\usepackage[TS1,T2A]{fontenc}
\usepackage[utf8]{inputenc}
\usepackage[english,russian]{babel}
\usepackage{tikz}
\usepackage[european,cuteinductors,smartlabels]{circuitikz}

\usepackage{amsmath,amsfonts}
\usepackage{amssymb}
\usepackage[scr]{rsfso}

\usepackage{gnuplottex}


%%% Межстрочный интервал
\usepackage{setspace}

%% таблицы
\usepackage{booktabs}

%% для кода
\usepackage{color}
\usepackage{listingsutf8}


\definecolor{lightgrey}{rgb}{0.9,0.9,0.9}
\definecolor{lightblue}{rgb}{0,0,1}

\definecolor{grey}{rgb}{0.5,0.5,0.5}
\definecolor{blue}{rgb}{0,0,1}
\definecolor{violet}{rgb}{0.5,0,0.5}

\definecolor{darkred}{rgb}{0.5,0,0}
\definecolor{darkblue}{rgb}{0,0,0.5}
\definecolor{darkgreen}{rgb}{0,0.5,0}


\lstset{%
  language=C++,%
  morekeywords={constexpr,nullptr,size_t,uint32_t,assert,override,final},%
  basicstyle=\ttfamily\footnotesize,%
  sensitive=true,%
  keywordstyle=\color{blue},%
  stringstyle=\color{darkgreen},%
  commentstyle=\color{violet},%
  showstringspaces=false,%
  tabsize=2,%
  frame=leftline,
  rulecolor=\color{lightblue},
  xleftmargin=10pt,
}

\lstset{
%extendedchars=\true,
%inputencoding=utf8x,   
  numberstyle=\tiny,
  numbers=left,
  numbersep=10pt,
  xleftmargin=10pt,
  %framesep=4.5mm,
  %framexleftmargin=2.5mm,
  framexleftmargin=5pt,
  framesep=15pt,
  fillcolor=\color{lightgrey},
}


\title{Повышающий преобразователь}

\begin{document}
\section{повышающий преобразователь}

\subsection{определение номиналов элементов схемы}
\begin{figure}[!ht]
\centering
\begin{tikzpicture}
        \draw(0,3) to[L,l=L,o-] (3,3) -- (6,3) to [C,l_=C] (6,0) to[short,-o] (0,0);
        \draw(4,3) to[short] (8.5,3) to[R,l_=R] (8.5,0) to [short] (6,0);
        \draw(3,0) to[cosw,l_=SW] (3,3);
        \draw (3,3) to[D-,l=D] (6,3);
        \draw[->,>=latex] (9.2,2.1) -- (9.2,0.9) node[midway,right] {$i_1$};
        \draw[->,>=latex] (6.7,2.1) -- (6.7,0.9) node[midway,right] {$i_2$};
        \draw (0,1.5) node {$U_\textcyrillic{пит.}$};
\end{tikzpicture}
\caption{повышающий преобразователь}
\end{figure}

$U_\textcyrillic{пит.}$ -- заданное значение 100В. Подводимая мощность 100Вт. Частота переключения ключа $SW$ 5кГц. 
$U_R = U_\textcyrillic{нагр.}$ -- выходное напряжение на нагрузке $R$ зависит от варианта:

\begin{tabular}{c|c}
\toprule
\textnumero п/п & $U_\textcyrillic{вых}$ (В) \\
\midrule
1 & 110 \\
2 &115 \\
3 & 120 \\
4 & 125 \\
5 & 130 \\ 
6 & 135 \\
7 & 140 \\
8 & 145 \\
9 & 150 \\
10 &155 \\
11 & 160 \\
12 & 165 \\
13 & 170 \\
14 & 175 \\
15 & 180 \\
16 & 185 \\
17 & 190 \\
18 &195 \\
19 & 200 \\
\bottomrule
\end{tabular}
 
Требуется определить номиналы элементов $R$, $L$, $C$ исходя из условий: 
\begin{itemize}
\item сила тока на дросселе $I_L(t)$ изменяется во времени не более чем
$I_{L\textcyrillic{ср.}}\pm 10$\%,
\item напряжение на нагрузке $U_R(t)$ изменяется во времени не более чем $U_{R\textcyrillic{ср.}}\pm 10$\%.  
\end{itemize}



Рассмотрим работу преобразователя в двух стадиях:

\subsection{фаза заряда  дросселя}

\begin{figure}[!ht]
\centering
\begin{tikzpicture}
\draw[thin, ->, >=latex] (0,0) -- (6,0) node[below] {$t$};
\draw[thin, ->, >=latex] (0,0) -- (0,4) node[left] {$i(t)$};
\draw (0.5,1.5) -- (1.5,3) -- (3,1.5) -- (4,3) -- (5.5,1.5);
\draw[dashed] (5.8,2.25) -- (0,2.25) node[left] {$I_\textcyrillic{ср.}$};
\draw[thin] (0.5,1.5) -- (0.5,-0.8) (1.5,3) -- (1.5,-0.8) (3,1.5) -- (3,-0.8);
\draw[thin,<->] (0.5,-0.5) -- (1.5,-0.5) node[midway, below] {$t_\textcyrillic{з}$};
\draw[thin,<->] (1.5,-0.5) -- (3,-0.5) node[midway, below] {$t_\textcyrillic{р}$};
\draw[thin] (0.5,-1.2) -- (0.5,-1.8)  (3,-1.2) -- (3,-1.8);
\draw[thin,<->] (0.5,-1.5) -- (3,-1.5) node[midway, below] {$T$};
\end{tikzpicture}
\end{figure}

В установившемся режиме
\begin{equation}
U_\textcyrillic{нагр.} = U_\textcyrillic{пит.}\frac{t_\textcyrillic{з} + t_\textcyrillic{р}}{ t_\textcyrillic{р}} =  U_\textcyrillic{пит.}\frac{T}{ t_\textcyrillic{р}}
\end{equation}

Ключ замкнут, дроссель соединен с землей, диод через себя не дает разрядиться конденсатору, конденсатор разряжается через нагрузку.

\begin{figure}[!ht]
\centering
\begin{tikzpicture}
\draw(0,2) to [L,o-] (2,2) -- (2,0) to[short,-o] (0,0);
\draw ({4+0},0) to[C] ({4+0},2) -- ({4+2},2) to[R] ({4+2},0) -- ({4+0},0);
\end{tikzpicture}
\caption{фаза заряда дросселя}
\end{figure}


напряжение на дросселе
$$
U_L = L\frac{\partial i_\textcyrillic{з}}{\partial t}
$$
или, интегрируя

$$
 i(t_\textcyrillic{з}) = i_{(0)} + \frac{U}{L} t_\textcyrillic{з}
$$

Принимая $i_{(0)}$ равным нулю (условие, когда наступает прерывистый режим) получаем что максимальный 
${\displaystyle i_{max} =  \frac{U}{L} t_\textcyrillic{з}}$.  
Ток меняется от 0 до $i_{max}$, значит 
\begin{equation}
i_\textcyrillic{среднее} = i_{max}/2 = 
\frac{U}{2L} t_\textcyrillic{з}
\end{equation}

или, иначе, ток меняется 

$$i(t) = i_\textcyrillic{среднее} \pm 100\%$$

Отсюда получаем минимальный $L$ необходимый для отсутствия прерывистого режима:
$$
L_{min} = \frac{U}{2} \frac{t_\textcyrillic{з}}{i_{max}}
$$

Средний ток получаем из заданной подводимой мощности 100Вт и напряжения батареи 100В:
$$
i_\textcyrillic{среднее}\cdot U = P
$$


Если предполагаем, что $i_L$ через дроссель меняется 
$$i(t) = i_\textcyrillic{среднее} \pm 10\%$$,

то индуктивность дросселя в 10 раз больше чем $L_{min}$



Номинал конденсатора определим исходя из того что за время $t_\textcyrillic{з}$ через нагрузку $R$ разряжается на 20\%
от максимального заряда (или заряд конденсатора меняется $U_\textcyrillic{нагр. ср.} \pm 10\%$). 


\subsection{фаза разряда дросселя} 
%\begin{figure}[!ht]
%\centering
\begin{tikzpicture}
        \draw(0,3) to[L,l=L,o-] (3,3) -- (4,3) to [C,l_=C] (4,0) to[short,-o] (0,0);
        \draw(4,3) to[short,*-] (6.5,3) to[R,l_=R] (6.5,0) to [short,-*] (4,0);
        \draw[->,>=latex] (7.2,2.1) -- (7.2,0.9) node[midway,right] {$i_1$};
        \draw[->,>=latex] (4.7,2.1) -- (4.7,0.9) node[midway,right] {$i_2$};
        \draw (0,1.5) node {$U_1$};
\end{tikzpicture}
%\caption{фаза разряда дросселя}
%\end{figure}

\subsection{Выполнение работы на виртуальной установке}

Определив номиналы выполняем работу следуя методичке "Исследование повышающего широтно-импульсного преобразователя постоянного напряжения"
на виртуальной установке.



\end{document}
