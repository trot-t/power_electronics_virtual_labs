\documentclass{article}
\usepackage[utf8]{inputenc}
\usepackage[english,russian]{babel}  % добавили русский
\usepackage{tikz}                    % черчение графиков
\usepackage{siunitx}                 % в электрических схемах номиналы элементов
\usepackage[american,cuteinductors,smartlabels]{circuitikz} % черчение электрических схем
\usepackage{hyperref}
\usepackage{booktabs}
\usepackage[left=10mm, top=20mm, right=10mm, bottom=20mm, nohead, footskip=7mm]{geometry} % настройки полей документа

\title{Практическая работа 1} % знак подчеркивание - символ построчного индекса, замаскируем его обратным слешом
\author{Прокшин А.Н.}
\date{April 2020}

\begin{document}

\maketitle

\section{Индивидуальное задание на практическую работу №1}

\newcommand{\PI}{3.14159265}
\newcommand{\Fi}{54.08/180*3.14159265} % угол между фазами K и K+1  54.08 градусов

Фаза $k+1$ отстает от фазы $k$ на угол $\varphi=360^\circ/m$, где $m$-пульсность схемы. Заданное значение угла $\varphi$ в градусах вносим в файл-исходник 
вместо значения параметра $\backslash Fi$, которое для примера выбрано  $54.08$:

\begin{verbatim}
\newcommand{\Fi}{54.08/180*3.14159265}
\end{verbatim}

Аналогично вносим значения заданных параметров $\alpha$ и $\gamma$ 
\newcommand{\alfa}{0.524}  % от оси отсчета отложим угол alpha, поскольку переменная \alpha уже есть, поменял имя переменной на \alfa 
\newcommand{\gammaa}{0.698} % угол gamma


найдем пересечение графиков, это точка, откуда будет отсчитываться угол управления $\alpha$, т.е. надо решить уравнение

\begin{equation} % начало нумерованной формулы
sin(\omega t) = sin(\omega t - \varphi)
	\label{eq1}
\end{equation}   % конец формулы



решаем уравнение \ref{eq1} в любом математическом пакете, например, в \href{http://www.reduce-algebra.com/obtaining.php}{reduce-algebra}, 
находим  x-координату точки пересечения графиков фаз (в моем случае 2.043). получено следующим образом:
\begin{verbatim}
on rounded; %reduce-algebra вычисляет в символьном виде, вычислить округленно
solve(sin(x) - sin(x-54.08/180*3.14159265),x);
\end{verbatim}

полученную значение вносим в файл на место значения параметра $\backslash x$:
\newcommand{\x}{2.043}

\begin{verbatim}
\newcommand{\x}{2.043}
\end{verbatim}

Аналогично вычисляем x-координату точки пересечения графиков фаз $k+1$ и $k+2$ для параметра $\backslash xI$ и  фаз $k+2$ и $k+3$ для параметра $\backslash xII$ 
и меняем значения этих параметров в файле.

\noindentпроводим линию начала отсчета углов управления из точки (x-коодината, sin(x-коодината)) вниз до точки с координатой $y=-2$

\begin{figure}[ht!]
\centering
\begin{circuitikz}
        \draw[thin,->] (0,0) -- (13,0) node[below] {$\omega t$}; %подпишем ось, греческие буквы как команды \alpha
	\draw[thin,->] (0,0) -- (0,2) node[left] {$U_{2\textcyrillic{ф}}$} ; % ось y

        \draw[domain=0:12, samples=200, help lines, smooth]              % samples - количество точек для гладкости графика
        plot ( \x, {sin(\x r)} ); % график 1
        \draw[domain=0:12, samples=200, help lines, smooth]   % help lines, smooth - стиль линии, можно добавить расскраску - red
        plot (\x, {sin((\x-\Fi) r)} );  % график 2
        
        \draw[thin] ({\x}, -2 ) -- ({\x},{sin(\x r)}); % проводим линию оси отсчета углов

	\draw[<->] ({3.14/2}, {1.4}) -- ({3.14*(1/2+54.08/180)},{1.4}) node[midway, above] {$360^\circ/m$};
\end{circuitikz}
\caption{две последовательные фазы с точкой отсчета в точке коммутации неуправляемых вентилей $\alpha=0$}
\end{figure}


\newcommand{\Fii}{\Fi} % угол между фазами K+1 и K+2  54.08 градусов

%новая точка пересечения solve(sin(x-54.08/180*3.14159265) - sin(x-54.08/180*3.14159265-54.08/180*3.14159265),x);
\newcommand{\xI}{2.987}

\newcommand{\Fiii}{\Fi} % угол между фазами K+2 и K+3  54.08 градусов

%новая точка пересечения solve(sin(x-54.08/180*3.14159265-54.08/180*3.14159265) - sin(x-54.08/180*3.14159265-54.08/180*3.14159265-54.08/180*3.14159265),x);
\newcommand{\xII}{3.930}

\begin{figure}[!ht]
\centering
\begin{circuitikz}
        \draw[thin,->] (0,0) -- (13,0) node[below] {$\omega t$}; %подпишем ось, греческие буквы как команды \alpha
        \draw[thin,->] (0,0) -- (0,2) node[left] {$U_{2\textcyrillic{ф}}$} ; % ось y

        % фазы K и K+1
        \draw[domain=0:12, samples=200, help lines, smooth]              % samples - количество точек для гладкости графика
        plot ( \x, {sin(\x r)} ); % график 1 (фаза K)
        \draw[domain=0:12, samples=200, help lines, smooth]   % help lines, smooth - стиль линии, можно добавить расскраску - red
        plot (\x, {sin((\x-\Fi) r)} );  % график 2 (фаза K+1)

        %полусумма фаз K и K+1
        \draw[domain=\x:{\x+\PI}, samples=200, help lines, smooth, dashed]   % help lines, smooth - стиль линии, можно добавить расскраску - red
        plot (\x, {(sin(\x r) +sin((\x-\Fi) r))/2} );  % полусумма штриховой линией

        % фаза K+2
       \draw[domain=0:12, samples=200, help lines, smooth]   % help lines, smooth - стиль линии, можно добавить расскраску - red
        plot (\x, {sin((\x-\Fi-\Fii) r)} );  % график 3 (фаза K+2)
	% фаза K+3
       \draw[domain=0:12, samples=200, help lines, smooth]   % help lines, smooth - стиль линии, можно добавить расскраску - red
        plot (\x, {sin((\x-\Fi-\Fii-\Fiii) r)} );

        % подписи к графикам
        \draw ({\PI/2},{sin((\PI/2) r)}) node[rotate=90,right] {k} ({\PI/2+\Fi},{sin((\PI/2) r)}) node[rotate=90,right] {k+1} 
	({\PI/2+\Fi+\Fii},{sin((\PI/2) r)}) node[rotate=90,right] {k+2}  ({\PI/2+\Fi+\Fii+\Fiii},{sin((\PI/2) r)}) node[rotate=90,right] {k+3};
        %полусумма фаз K+1 и K+2
        \draw[domain=\xI:{\xI+\PI}, thin, smooth, dotted]
        plot (\x, {(sin((\x-\Fi) r) + sin((\x-\Fi-\Fii) r))/2} );


        \draw[thin] ({\x}, -2 ) -- ({\x},{sin(\x r)}); % проводим линию оси отсчета углов

        \draw[thin] ({\x + \alfa}, -2 ) -- ({\x + \alfa}, { (sin((\x+\alfa) r) +  sin((\x+\alfa - \Fi) r))/2 }); % линия для \alpha

        \draw[thin] ({\x + \alfa + \gammaa}, -2 ) -- ({\x + \alfa + \gammaa }, { sin((\x+\alfa+\gammaa - \Fi) r) }); % линия для \alpha + \gamma

        \draw[thin] ({\x+\PI}, -2 ) -- ({\x+\PI},{sin((\x+\PI) r)}); %

        %вертикальная жирная красная линия  для x-координаты соответствующий углу управления \x + \alpha  от графика фазы K  до полусуммы фаз K и K+1
        \draw[ultra thick, red]
        ({\x + \alfa}, { sin((\x+\alfa) r) }) -- ({\x + \alfa}, { (sin((\x+\alfa) r) +  sin((\x+\alfa - \Fi) r))/2 });  % x-координата соответствующая углу управления \x + \alpha
        % жирной красной линией продолжим вдоль графика полусуммы от \alpha до \alpha + \gamma
        \draw[domain=\x+\alfa:\x+\alfa+\gammaa, ultra thick, red] % от \alpha до \alpha + \gamma
                plot ({\x}, { (sin((\x) r) +  sin((\x - \Fi) r))/2 }); % вдоль полусуммы фаз K и K+1
        % вертикальная красная линия для x-координаты равной \x + \alpha + \gamma от графика полусуммы фаз K и K+1 до графика фазы K+1
        \draw[ultra thick, red]
        ({\x + \alfa + \gammaa}, { (sin((\x+\alfa+\gammaa) r) +  sin((\x+\alfa+\gammaa - \Fi) r))/2 }) --  ({\x + \alfa + \gammaa}, { sin((\x+\alfa+\gammaa - \Fi) r) });

	% линии вниз
        \draw[thin,dotted] ({\xI}, -2 ) -- ({\xI},{sin((\xI-\Fi) r)}); % проводим линию оси отсчета углов для фаз K+1 и K+2
        \draw[thin,dotted] ({\xI + \alfa}, -2 ) -- ({\xI + \alfa}, { (sin((\xI+\alfa-\Fi) r) +  sin((\xI+\alfa-\Fi-\Fii) r))/2 }); % линия для \alpha отсчитываемого от пересечения K+1 и K+2
        \draw[thin,dotted] ({\xI + \alfa + \gammaa}, -2 ) -- ({\xI + \alfa + \gammaa }, { sin((\xI+\alfa+\gammaa - \Fi - \Fii) r) }); % линия для \xI + \alpha + \gamma

        \draw[domain={\x + \alfa + \gammaa}:{\xI+\alfa},ultra thick, red] % вдоль графика фазы K+1 от конца коммутации до угла управления для фазы K+1
        plot(\x,{sin((\x-\Fi) r)});

       % и далее повторяем то что было для фаз K и K+1
        %вертикальная жирная красная линия  для x-координаты соответствующий углу управления \xI + \alpha  от графика фазы K+1  до полусуммы фаз K+1 и K+2
        \draw[ultra thick, red]
        ({\xI + \alfa}, { sin((\xI+\alfa-\Fi) r) }) -- ({\xI + \alfa}, { (sin((\xI+\alfa-\Fi) r) +  sin((\xI+\alfa-\Fi-\Fii) r))/2 }); % x-координата угла управления фазы K+1 \xI + \alpha
        % жирной красной линией продолжим вдоль графика полусуммы фаз K+1 и K+2   от \xI+\alpha до \xI+\alpha + \gamma
        \draw[domain=\xI+\alfa:\xI+\alfa+\gammaa, ultra thick, red] % от \xI+\alpha до \xI + \alpha + \gamma
                plot ({\x}, { (sin((\x-\Fi) r) +  sin((\x-\Fi-\Fii) r))/2 }); % вдоль полусуммы фаз K+1 и K+2
        % вертикальная красная линия для x-координаты равной \xI + \alpha + \gamma от графика полусуммы фаз K+1 и K+2 до графика фазы K+2
        \draw[ultra thick, red]
        ({\xI + \alfa + \gammaa}, { (sin((\xI+\alfa+\gammaa-\Fi) r) +  sin((\xI+\alfa+\gammaa-\Fi-\Fii) r))/2 }) --  ({\xI + \alfa + \gammaa}, { sin((\xI+\alfa+\gammaa-\Fi-\Fii) r) });

       % линии вниз
        \draw[very thin,loosely dashed] ({\xII}, -2 ) -- ({\xII},{sin((\xI-\Fi-\Fii) r)}); % проводим линию оси отсчета углов для фаз K+2 и K+3
        \draw[very thin,loosely dashed] ({\xII+\alfa},-2)--({\xII+\alfa},{(sin((\xII+\alfa-\Fi-\Fii) r)+sin((\xII+\alfa-\Fi-\Fii-\Fiii) r))/2 }); % линия для \alpha отсчитываемого от пересечения K+2 и K+3
        \draw[very thin,loosely dashed] ({\xII+\alfa+\gammaa},-2)--({\xII+\alfa+\gammaa},{sin((\xII+\alfa+\gammaa-\Fi-\Fii-\Fiii) r)}); % линия для \xII + \alpha + \gamma

	% и далее повторяем то что было для фаз K+1 и K+2
	\draw[domain={\xI + \alfa + \gammaa}:{\xII+\alfa},ultra thick, red] % вдоль графика фазы K+2 от конца коммутации до угла управления для фазы K+2
        plot(\x,{sin((\x-\Fi-\Fii) r)});
	%вертикальная жирная красная линия  для x-координаты соответствующий углу управления \xII + \alpha  от графика фазы K+2  до полусуммы фаз K+2 и K+3
        \draw[ultra thick, red]
        ({\xII + \alfa},{sin((\xII+\alfa-\Fi-\Fii) r)})--({\xII+\alfa},{(sin((\xII+\alfa-\Fi-\Fii) r)+ sin((\xII+\alfa-\Fi-\Fii-\Fiii) r))/2});%x-координата угла управления фазы K+2 \xII+\alpha
	% жирной красной линией продолжим вдоль графика полусуммы фаз K+2 и K+3   от \xII+\alpha до \xII+\alpha + \gamma
        \draw[domain=\xII+\alfa:\xII+\alfa+\gammaa, ultra thick, red] % от \xII+\alpha до \xII + \alpha + \gamma
	plot ({\x}, { (sin((\x-\Fi-\Fii) r) +  sin((\x-\Fi-\Fii-\Fiii) r))/2 }); % вдоль полусуммы фаз K+2 и K+3
        % вертикальная красная линия для x-координаты равной \xII + \alpha + \gamma от графика полусуммы фаз K+2 и K+3 до графика фазы K+3
        \draw[ultra thick, red]
        ({\xII+\alfa+\gammaa},{(sin((\xII+\alfa+\gammaa-\Fi-\Fii) r)+sin((\xII+\alfa+\gammaa-\Fi-\Fii-\Fiii) r))/2})--({\xII+\alfa+\gammaa},{ sin((\xII+\alfa+\gammaa-\Fi-\Fii-\Fiii) r) });	
\end{circuitikz}
\caption{коммутация между фазами $k$ и $k+1$, и %коммутация 
между фазами $k+1$ и $k+2$, между фазами $k+2$ и $k+3$ }
\label{answer}
\end{figure}

На получившемся графике обозначаем в виде стрелочки $\leftrightarrow$ угол управления $\alpha$, угол коммутации $\gamma$, угол $\beta$.
Пример вычерчивания стрелки приведен этом файле
\begin{verbatim}
\draw[<->] ({3.14/2}, {1.4}) -- ({3.14},{1.4}) node[midway, above] {$\alpha$};
\end{verbatim}

Данный пример доступен \href{https://www.overleaf.com/read/jrvwwzycpzvp}{по адресу}:
\begin{verbatim}
https://www.overleaf.com/read/jrvwwzycpzvp
\end{verbatim}

\begin{table}[ht!]
\centering
\begin{tabular}{l|c|c|c}
\toprule
	\textnumero п/п&	$m$-пульсность схемы & угол управления $\alpha$ & угол коммутации $\gamma$ \\
\midrule
	1&	3 & 30 & 20\\ 
	2&	3 & 35 & 90\\ 
	3&	3 & 45 & 15\\ 
	4&	3 & 60 & 40\\ 
	5&	3 & 90 & 40\\ 
	6&	3 & 110 & 10\\ 
	7&	3 & 130 & 10\\ 
        8&      6 & 30 & 20\\
        9&      6 & 45 & 15\\
        10&      6 & 60 & 40\\
        11&      6 & 90 & 40\\
        12&      6 & 110 & 10\\
        13&      6 & 130 & 10\\
        14&      12 & 30 & 20\\
        15&      12 & 45 & 15\\
        16&      12 & 60 & 40\\
        17&      12 & 90 & 40\\
        18&      12 & 110 & 10\\
        19&      12 & 130 & 10\\
\bottomrule
\end{tabular}
	\caption{индивидуальные задания}
\end{table}
\end{document}

