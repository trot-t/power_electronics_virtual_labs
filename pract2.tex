\documentclass{article}
\usepackage[T2A]{fontenc}
\usepackage[utf8]{inputenc}
\usepackage[english,russian]{babel}
\usepackage{tikz}
\usepackage[european,cuteinductors,smartlabels]{circuitikz}
\title{Практическая работа №2}
%\author{студент:    группы    }
% Конец преамбулы
\begin{document}
\maketitle

\begin{figure}[!ht]
\begin{tabular}{cccc}
\begin{minipage}{0.25\textwidth}
	\includegraphics[scale=0.3]{schema1}
	\caption{\small Лучевая схема: треугольник -- звезда}
\end{minipage}
	&
\begin{minipage}{0.25\textwidth}
        \includegraphics[scale=0.3]{schema2}
	\caption{\small Лучевая схема: звезда -- звезда}
\end{minipage}
        &
\begin{minipage}{0.3\textwidth}
        \includegraphics[scale=0.3]{schema3}
	\caption{\small Мостовая схема (Ларионова): треугольник -- звезда}
\end{minipage}
        &
\begin{minipage}{0.25\textwidth}
        \includegraphics[scale=0.3]{schema4}
	\caption{\small Мостовая схема (Ларионова): звезда -- звезда}
\end{minipage}
       \\
\begin{minipage}{0.25\textwidth}
        \includegraphics[scale=0.3]{schema5}
	\caption{\small последовательная схема (Вологдина): треугольник -- две звезды}
\end{minipage}
        &
\begin{minipage}{0.25\textwidth}
        \includegraphics[scale=0.3]{schema6}
	\caption{\small последовательная схема (Вологдина): звезда -- две звезды}
\end{minipage}
        &
\begin{minipage}{0.25\textwidth}
        \includegraphics[scale=0.3]{schema7}
	\caption{\small Мостовая схема: треугольник -- треугольник}
\end{minipage}
        &
\begin{minipage}{0.25\textwidth}
        \includegraphics[scale=0.3]{schema8}
	\caption{\small Мостовая схема: звезда -- треугольник}
\end{minipage}
       \\
\begin{minipage}{0.3\textwidth}
        \includegraphics[scale=0.3]{schema9}
	 \caption{\small Параллельная схема (Кюблера): треугольник -- две звезды}
\end{minipage}
        &
\begin{minipage}{0.3\textwidth}
        \includegraphics[scale=0.3]{schema10}
	\caption{\small Параллельная схема (Кюблера): звезда -- две звезды}
\end{minipage}
        &
\begin{minipage}{0.3\textwidth}
        \includegraphics[scale=0.3]{schema11}
	\caption{\small Лучевая схема: треугольник -- двойной зигзаг}
\end{minipage}
        &
\begin{minipage}{0.3\textwidth}
        \includegraphics[scale=0.3]{schema12}
	\caption{\small Лучевая схема: звезда -- двойной зигзаг}
\end{minipage}
       \\
\end{tabular}
\end{figure}


\section*{Задание}
\begin{itemize}
	\item Изобразить схему согласно варианта на рис 1-12 (схема должны быть редактируемой в Компасе или Latex, элементы схемы в векторном виде -- при изменении размера изображение не должно деградировать);
	\item изобразить вектора фаз первичной обмотки трансформатора, вторичной обмотки трансформатора для выбранного момента времени;
	\item построить график выпрямленного напряжения для одного периодa.

\end{itemize}


%Пример графика:

%\begin{figure}[ht!]
%\centering
%\begin{tikzpicture}[scale=0.67]
%\newcommand{\xb}{-3} % введем переменную \xb равную -3
%\newcommand{\xa}{3}
%\draw[thin, ->] (-6,0) -- (6,0) node[right] {$X$}; % чертим ось x
%\draw[thin, ->] (0,-1.5) -- (0,1.5) node[left] {$Y$};  % чертим ось y
% % подписи под точками на оси
%\foreach \x\xtext in {-5/-5,5/5,{\xb}/\xb,{\xa}/{\gamma}} %
%   \draw (\x,0.1) -- (\x,-0.1) node[below] {$\xtext$};
% % строим график
% \draw[domain=-5:0, help lines, smooth] % график для интервала -5:0
%        plot ({\x},{sin(\x*180/3.14)}); % фунция sin имеет аргумент в градусах
% \draw[domain=0:5, help lines, smooth]  % график для интервала 0:5
%        plot ({\x},{cos(\x*180/3.14)});
%\end{tikzpicture}
%	\caption{Пример вычерчивания графика}
%\end{figure}
\end{document}
